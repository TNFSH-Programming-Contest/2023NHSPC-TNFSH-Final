\PassOptionsToPackage{dvipsnames}{xcolor}
\documentclass[hyperref,UTF8,notheorems,xcolor={dvipsnames}]{beamer}
\usepackage[dvipsnames]{xcolor}
\usetheme{Rochester}
\usepackage{amsmath, mathtools, graphicx}
\usepackage{standalone}
\usepackage[most]{tcolorbox}
\usepackage[normalem]{ulem}
\usepackage{xeCJK}
\usepackage{fontspec}
\usepackage{tikz}
\usepackage{pgfplots}
\usepackage{pifont}
\usepackage{fontawesome5}
\usepgfplotslibrary{polar}
\usepgflibrary{shapes.geometric}
\usetikzlibrary{calc,angles,quotes}
\defaultfontfeatures{Mapping=tex-text}
\usefonttheme{professionalfonts}
\usepackage{concmath}
\usepackage{minted}

\newcommand{\cmark}{\ding{51}}%
\newcommand{\xmark}{\ding{55}}%

%\newcommand{\CC}[1]{\textsf{\small #1}}% to quote from Computing Curricula 2001
\newcommand{\CC}[1]{#1}

\newcommand{\Cincluded}{{\small\cmark}}
\newcommand{\Cdefine}{{\small\cmark\faFile*[regular]}}
\newcommand{\Ccodeonly}{{\small\cmark\faFile*~}}
\newcommand{\Cnofocus}{{\small\faQuestion}}
\newcommand{\Cexmaybe}{{\small\xmark\faQuestionCircle}}
\newcommand{\Cexcluded}{{\small\xmark}}

\newcommand{\Iincluded}{\item[\hbox to 1.8em{\Cincluded\hfill}]}
\newcommand{\Idefine}{\item[\hbox to 1.8em{\Cdefine\hfill}]}
\newcommand{\Icodeonly}{\item[\hbox to 1.8em{\Ccodeonly\hfill}]}
\newcommand{\Inofocus}{\item[\hbox to 1.8em{\Cnofocus\hfill}]}
\newcommand{\Iexmaybe}{\item[\hbox to 1.8em{\Cexmaybe\hfill}]}
\newcommand{\Iexcluded}{\item[\hbox to 1.8em{\Cexcluded\hfill}]}

\DeclareRobustCommand{\rddots}{\text{\reflectbox{$\ddots$}}}


\usepackage{graphicx}
\graphicspath{ {./../images/} }

\def\codesize{\fontsize{8}{9}\selectfont}
\setmonofont[Mapping=]{Source Code Pro}
\setminted{fontsize=\codesize, linenos, frame=lines, mathescape, autogobble, tabsize=4}
\setCJKmainfont[AutoFakeSlant,BoldFont=Noto Sans CJK TC Bold]{Noto Sans CJK TC}

\setlength{\parskip}{\baselineskip} 
\newcommand{\btitle}[1]{{\secname} -- #1}

\theoremstyle{definition}
\newtheorem{theorem}{定理}
\newtheorem{lemma}{引理}
\newtheorem{property}{性質}
\newtheorem{corollary}{推論}
\newtheorem{problem}{例題}


\newtheorem{definition}{定義}
\AtBeginEnvironment{definition}{
  \setbeamercolor{block title}{fg=white,bg=red!70!black}
  \setbeamercolor{block body}{fg=black, bg=block title.bg!10!bg}
}

\newtheorem{exercise}{習題}
\AtBeginEnvironment{exercise}{
  \setbeamercolor{block title}{fg=white,bg=green!30!black}
  \setbeamercolor{block body}{fg=black, bg=block title.bg!10!bg}
}


\AtBeginSection[]{
  \begin{frame}
    \tableofcontents[currentsection,hideallsubsections]
  \end{frame}
  \begin{frame}
  \vfill
  \centering
  \begin{beamercolorbox}[sep=6pt,center,shadow=true,rounded=true]{title}
    \usebeamerfont{title}\LARGE\insertsectionhead\par%
  \end{beamercolorbox}
  \vfill
  \end{frame}
}


\AtBeginSubsection[]{
  \begin{frame}
    \tableofcontents[subsectionstyle=show/shaded/hide]
  \end{frame}
}

\usepackage{ctable}
\usepackage{tabularx}

\setlength{\parskip}{\baselineskip}


\title{112 臺南一中學科能力競賽校內複選}

\hypersetup{CJKbookmarks=true}
\begin{document}

\author{題解}
\date{Sep 28 2023}

\begin{frame}
  \titlepage
\end{frame}

\section*{Overview}

\begin{frame}[fragile]{\btitle{預期解出人數}}
	\begin{center}
		\begin{tikzpicture}
			\begin{axis} [%
				ybar,
				bar width=15pt,
				xmin=0.5,
				xmax=5.5,
				ymin=0,
				xtick={1,2,3,4,5},
				xticklabels={A,B,C,D,E}]
			\addplot plot coordinates {
				(1,5) 
				(2,3) 
				(3,0) 
				(4,1)
				(5,10)
			};
			\addplot [color=white] plot coordinates {
				(1,1) 
				(2,1) 
				(3,1) 
				(4,1)
				(5,1)
			};
			\end{axis}
		\end{tikzpicture}
	\end{center}
\end{frame}

\begin{frame}[fragile]{\btitle{實際解出人數}}
	\begin{center}
		\begin{tikzpicture}
			\begin{axis} [%
				ybar,
				bar width=15pt,
				xmin=0.5,
				xmax=5.5,
				ymin=0,
				xtick={1,2,3,4,5},
				xticklabels={A,B,C,D,E}]
			\addplot plot coordinates {
				(1,5) 
				(2,3) 
				(3,0) 
				(4,1)
				(5,10)
			};
			\addplot [color=white] plot coordinates {
				(1,1) 
				(2,1) 
				(3,1) 
				(4,1)
				(5,1)
			};
			\end{axis}
		\end{tikzpicture}
	\end{center}
\end{frame}

\begin{frame}[fragile]{\secname}
	\begin{enumerate}
		\item A 圖論 + 一些 case
		\item B 資料結構
		\item C 數學
		\item D DP
		\item E 簽到題
	\end{enumerate}
\end{frame}

\section{E. 宗教戰爭 (religion)}

\begin{frame}[fragile]{\btitle{題目敘述}}
	給定由 \texttt{b, p, d, q} 組成的字串,問是否能夠在數次旋轉或翻轉之後一樣。  
\end{frame}

\begin{frame}[fragile]{\btitle{子任務}}
	\begin{enumerate}
		\item 字串長度為 1
		\item 無額外限制
	\end{enumerate}
\end{frame}

\begin{frame}[fragile]{\btitle{子任務 1 --- 字串長度為 1}}
	無論如何都是 \texttt{Yes}。
\end{frame}

\begin{frame}[fragile]{\btitle{子任務 2 --- 無額外限制}}
	旋轉:\texttt{b} 與 \texttt{q} 互換、\texttt{d} 與 \texttt{p} 互換並反轉順序

	上下翻轉:\texttt{b} 與 \texttt{p} 互換、\texttt{d} 與 \texttt{q} 互換
	
	左右翻轉:旋轉與上下翻轉。  

	只有四種可能,全部暴力檢查。  
\end{frame}

\section{A. 網路連線 (connection)}

\begin{frame}[fragile]{\btitle{題目敘述}}
	給定一張簡單圖,問有多少加 $k$ 條邊的組合滿足

	\begin{itemize}
		\item 圖連通
		\item 圖仍為簡單
	\end{itemize}
\end{frame}

\begin{frame}[fragile]{\btitle{子任務}}
	\begin{enumerate}
		\item $k = 1$
		\item $N \leq 20$
		\item $N \leq 160$
		\item 無額外限制
	\end{enumerate}
\end{frame}

\begin{frame}[fragile]{\btitle{子任務 1 --- $k = 1$}}
	假設圖的連通塊只有 $1$ 個,答案是 $\frac{N(N-1)}{2} - M$。\\
	小心 $\frac{N(N-1)}{2} \approx 3.2 \times 10^9$,注意 overflow。
	
	假設圖的連通塊有 $2$ 個,大小分別為 $c_1, c_2$,答案是 $c_1 \times c_2$。

	假設圖的連通塊有 $3$ 個以上,不可能滿足,答案為 $0$。
\end{frame}

\begin{frame}[fragile]{\btitle{子任務 2 --- $N \leq 20$}}
	枚舉所有邊的組合:$O(N^4)$

	求連通狀態:DFS $O(N)$

	總時間複雜度 $O(N^5)$。  
\end{frame}

\begin{frame}[fragile]{\btitle{子任務 3 --- $N \leq 160$}}
	假設圖的連通塊只有 $1$ 個,答案是 $\frac{1}{2}\left(\frac{N(N-1)}{2} - M\right)\left(\frac{N(N-1)}{2} - M - 1\right)$。
	

	假設圖的連通塊有 $2$ 個,大小分別為 $c_1, c_2$,\\
	考慮有兩條連接兩個連通塊的邊,是 $\frac{1}{2}(c_1 \times c_2)(c_1 \times c_2 - 1)$。\\
	有一個連通兩個連通塊的邊,另一個不是:$c_1\times c_2 \times (M - \frac{1}{2}c_1(c_1-1) - \frac{1}{2}c_2(c_2-1))$。  


	假設圖的連通塊有 $3$ 個,大小分別為 $c_1, c_2, c_3$,答案為 $(c_1 + c_2 + c_3)c_1c_2c_3$。  
	

	假設圖的連通塊有 $3$ 個以上,不可能滿足,答案為 $0$。 
\end{frame}

\begin{frame}[fragile]{\btitle{子任務 4 --- 無額外限制}}
	$\frac{1}{2}\left(\frac{N(N-1)}{2} - M\right)\left(\frac{N(N-1)}{2} - M - 1\right) \approx 5 \times 10^{18}$  

	善用 \texttt{(unsigned) long long}!
\end{frame}

\section{B.}

\section{D. 森林道路 (pathway)}

\begin{frame}[fragile]{\btitle{題目敘述}}
	給定一個 $N \times M$ 的網格,求出最大權的道路。  

	道路的條件是:
	\begin{itemize}
		\item $(1, 1), (N, M)$ 要在道路上
		\item 對於任意兩個左上 -- 右下分布的兩個格子,道路要包含他們的最短路。
	\end{itemize}
\end{frame}

\begin{frame}[fragile]{\btitle{子任務}}
	\begin{enumerate}
		\item 只需要驗證輸入是不是合法的道路
		\item $NM \leq 50$
		\item $NM \leq 2000$
		\item $NM \leq 10^5$
	\end{enumerate}
\end{frame}

\begin{frame}[fragile]{\btitle{子任務 1 --- 驗證}}
	
	首先 $(1, 1)$ 至 $(N, M)$ 連通。  

	\onslide<2->
	{
		對於同一列的兩個格子,中間的格子都必須存在。\\  
		對於同一行的兩個格子,中間的格子都必須存在。  
	}

	\onslide<3->
	{
		假設相鄰兩列的格子是 $(i, L_i)$ 至 $(i, R_i)$ 以及 $(i+1, L_{i+1})$ 至 $(i+1, R_{i+1})$,則  

		$L_i \leq L_{i+1}$,否則 $(1, 1)$ 至 $(i+1, L_{i+1})$ 沒有最短路被包含。\\
		$R_i \leq R_{i+1}$,否則 $(i, R_i)$ 至 $(N, M)$ 沒有最短路被包含。\\
		$R_i \leq L_{i+1}$,否則不連通。
	}

\end{frame}

\begin{frame}[fragile]{\btitle{子任務 1 --- 驗證}}
	
	這樣的條件就合法嗎?  

	容易驗證最短路的條件一定滿足。  

	複雜度 $O(NM)$。  

\end{frame}

\begin{frame}[fragile]{\btitle{子任務 2 --- $NM \leq 50$}}
	
	定義 $dp_{i, l, r}$ 表示考慮到第 $i$ 列取了 $[l, r]$ 的所有格子,最大的權重可能。  

	初始值 $dp_{0, 1, 1} = 0$ 而剩餘的狀態都是 $- \infty$。  

	轉移直接按照剛剛的條件並枚舉,最後答案是 $\max_{l} dp_{n,l,m}$。  

	總時間複雜度 $O(NM^2 \times M^2)$。  

\end{frame}

\begin{frame}[fragile]{\btitle{子任務 3 --- $NM \leq 2000$}}
	
	$dp_{i, l, r}$ 的轉移來源:
	
	\[ \max_{\substack{1 \leq l' \leq l \\ l \leq r' \leq r}} dp_{i-1, l', r'} + \sum_{j=l}^{r}a_{i,j} \]
	
	當 $i, l$ 固定的時候,能夠轉移的上一個狀態左界 $1 \leq l' \leq l$ 是固定的,右界依序遞增。\\
	因此我們只需要對 $l'$ 預處理前綴最大值,轉移的時候逐漸加入 $r$ 即可只花 $O(1)$ 轉移。

	總時間複雜度 $O(NM^2)$。  

\end{frame}

\begin{frame}[fragile]{\btitle{子任務 4 --- $NM \leq 10^5$}}
	
	\[ \min(N, M) \leq \sqrt{NM} \]

	假如 $M > N$ 將網格轉置,可以發現條件是相同的,所以直接做一樣的 DP。  

	總時間複雜度 $O(\min(NM^2, N^2M)) = O(NM\min(N, M)) = O(NM\sqrt{NM})$。  

\end{frame}

\section{C. 老舊鍵盤 (keyboard)}

\begin{frame}[fragile]{\btitle{題目敘述}}

	找到最小的\textbf{正整數} $M$ 使得 $N$ 整除 $\overset{M}{\overbrace{11\cdots1}}$。
\end{frame}

\begin{frame}[fragile]{\btitle{子任務}}
	\begin{enumerate}
		\item $N \leq 10$
		\item $N \leq 2 \times 10^5$
		\item $N \leq 10^9$
	\end{enumerate}
\end{frame}

\begin{frame}[fragile]{\btitle{子任務 1 --- $N \leq 10$}}
	
	手算,答案表:

	\begin{tabular}{c|c|c|c|c|c|c|c|c|c|c}
	$N$ & 1 & 2 & 3 & 4 & 5 & 6 & 7 & 8 & 9 & 10 \\
	\hline
	$M$ & 1 & -1 & 3 & -1 & -1 & -1 & 6 & -1 & 9 & -1
	\end{tabular}

\end{frame}

\begin{frame}[fragile]{\btitle{子任務 2 --- $N \leq 2 \times 10^5$}}
	
	我們沒有辦法計算太大的數字,要怎麼搜索?

	\onslide<2->
	{
		\[ \overset{k+1}{\overbrace{11\cdots1}} = 10 \cdot \overset{k}{\overbrace{11\cdots1}} + 1 \]

		定義 $f(x) \equiv 10 x + 1 \mod N$,我們想知道 $0$ 套幾次 $0$ 會變回 $0$。  

		搜索到 $N$ 以下就能找到答案(或者無解)了,為什麼?  
	}

\end{frame}

\begin{frame}[fragile]{\btitle{子任務 3 --- $N \leq 10^9$}}
	
	\begin{lemma}[無解條件]
		如果 $N$ 是 $2$ 或 $5$ 的倍數則無解。  
	\end{lemma}

	考慮尾數一直都是 $1$,所以這件事情顯然會無解。  

\end{frame}

\begin{frame}[fragile]{\btitle{子任務 3 --- $N \leq 10^9$}}
	
	考慮由 $0, 1, \ldots, N - 1$,當這些數字 $x \rightarrow x + 1$ 的時候,不難發現是一對一對應,也就是說他是一個排列。  

	考慮由 $0, 1, \ldots, N - 1$,當這些數字 $x \rightarrow 10x$ 的時候,也是一對一對應,因為:\\
	假設 $10a \equiv 10b \mod N$,則 $N | 10(a - b)$,因為 $\mathrm{gcd}(N, 10) = 1$ 所以 $N | (a - b)$ 但不可能發生。  
	
	所以說,$f(x) \equiv 10 x + 1 \mod N$ 是兩個排列的合成函數,仍然是一個排列,在這些數字跟 $f$ 作為有向邊構成的圖每個數字 in, out degree 都是 1,是由一堆環組成的,因此答案一定 $\leq N$。  

\end{frame}

\begin{frame}[fragile]{\btitle{子任務 3 --- $N \leq 10^9$}}
	
	找環長度可以由下列的作法在 $O(\sqrt N \log N)$ 內做到:  

	假設 $x_0 = 0, x_i = f(x_{i-1})$,先找到 $x_0, x_1, \cdots, x_{K-1}$。  

	因為 $f(x) = 10 x + 1$,所以 $\overset{K}{\overbrace{f(f(f(f(\cdots f}}(x) \cdots))))$ 也可以被寫成一個線性函數,也就是 $x_{i+K} = a x_i + b$。

	這時候可以一直計算 $x_{K}, x_{2K}, x_{3K} \cdots, x_{\lceil\frac N K\rceil K}$,一定會有其中一個與 $x_0, x_1, \cdots, x_{K-1}$ 對應到,當找到就能夠計算環的長度 $= iK - j$。  

	開一個 \texttt{map} 紀錄 $x_0, x_1, \cdots, x_{K-1}$,所以總時間複雜度是 $O((K + \frac N K)\log N)$,取 $K = O(\sqrt N)$ 有 $O(\sqrt N \log N)$。   

\end{frame}

\begin{frame}[fragile]{\btitle{子任務 3 --- $N \leq 10^9$}}
	
	Alternative Solution: 考慮
	\[ \overset{M}{\overbrace{11\cdots1}} \equiv 0 \mod N \Rightarrow \overset{M}{\overbrace{99\cdots9}} \equiv 0 \mod 9N \Rightarrow 10^M \equiv 1 \mod 9N \]

	因為 
	\[ 10^{\phi(9N)} \equiv 1 \mod 9N \]

	所以只需要對 $9N$ 因式分解求出 $\phi(9N)$,接著對 $\phi(9N)$ 的質因數 $p$ 不斷嘗試 $\frac{\phi(9N)}{p}$ 是否是更好的解就好,使用 Pollard Rho 可以做到 $O(\sqrt[4]N)$。  
	
\end{frame}


\end{document}

